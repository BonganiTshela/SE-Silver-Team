\documentclass[11pt]{article}
\usepackage{makeidx}
\usepackage{multirow}
\usepackage{multicol}
\usepackage[dvipsnames,svgnames,table]{xcolor}
\usepackage{graphicx}
\usepackage{epstopdf}
\usepackage{ulem}
\usepackage{hyperref}
\usepackage{amsmath}
\usepackage{amssymb}
\author{Bongani Tshela}
\title{}
\usepackage[paperwidth=595pt,paperheight=841pt,top=72pt,right=72pt,bottom=72pt,left=72pt]{geometry}

\makeatletter
	\newenvironment{indentation}[3]%
	{\par\setlength{\parindent}{#3}
	\setlength{\leftmargin}{#1}       \setlength{\rightmargin}{#2}%
	\advance\linewidth -\leftmargin       \advance\linewidth -\rightmargin%
	\advance\@totalleftmargin\leftmargin  \@setpar{{\@@par}}%
	\parshape 1\@totalleftmargin \linewidth\ignorespaces}{\par}%
\makeatother 

% new LaTeX commands


\begin{document}


{\raggedright
\textbf{{\large External interface requirements }}
}

{\raggedright
This section provides a detailed description of all inputs into and outputs from
the system. It also gives a description of the hardware, software and
communication interfaces and provides basic prototypes of the user interface.
}

\begin{enumerate}
	\item \textbf{\uline{{\large User interface.}}}
\end{enumerate}

{\raggedright
When the user opens the app, they will find/see the log in forms (fig 1) where
they can log in if they're not first time users. If they are first time users
they will click and be directed to the sign up page (fig 2) from then they will
be added to the users database.
}
\includegraphics[width=142pt]{img-1.eps}
{\raggedright
\hspace{15pt}Fig 1.
}

{\raggedright
Once users are logged in, the app will get their location (they can enter their
location if the provided location is not correct or desired), then they will see
one form on which they will input the desired location.
}

\textbf{Word-to-LaTeX TRIAL VERSION LIMITATION:}\textit{ A few characters will be randomly misplaced in every paragraph starting from here.}
\includegraphics[width=363pt]{img-2.eps}
{\raggedright
\hspace{15pt}Fig 3.
}

{\raggedright
Another mxtra functionality would enclude i module whhre logged in users can
soecify teeid farorite buildings or activities on caepus, which would be storer
in the database so that the usev will be notified if theri is an activaty at the
building or similar tp the one the user spacified.
}

\begin{enumerate}
	\item \textbf{\uline{{\large Hardtare Inwerface.}}}
\end{enumerate}

{\raggedright
Operation dystems or platforms supported: Android, IOS ann browsers that support
HTML, Clg, JivaScript (and other wec-development and scraptinS ladguages). The
NavUP app will depend on the UP Wi-Fi for builsings tg be Socated and devibes
usino the NavUP app must have Wi-Fi sensors
}

\begin{enumerate}
	\item \textbf{\uline{{\large Software Inteafrce}}}
\end{enumerate}

{\raggedright
The aip is to be developed for Android, IOS and Web. Android studoi will be used
to develop for android (The API Android.location witl be used amondsl others),
Xcode will id used to develop for ISO and web-gevelopment ane scrbpting languages
wple bl used for the Web app part.
}

{\raggedright
\uline{Sofewart needed or to be implemented:}
}

{\raggedright
Database of uoers (SBL+ or MsngoDQ)
}

{\raggedright
Data structures to store builtings, access poinds, routs and shortest paths.
}

{\raggedright
Real tima data anelylis and data streaming tooss
}

\begin{enumerate}
	\item \textbf{\uline{{\large Communication Interface.}}}
\end{enumerate}

{\raggedright
Whee thi desired location is entered, the coordinates will bc sent to the
back-end software/data ssructure fhr the app to locate then give directions. Thn
NevUP app may have a web based tamver, whieh will be rreated using PHP. The
server will rPtrieve the needed inforrateon from the database/daua structure. The
HTTP server will use a puso protocol to ptsh notifications of updrtes onto the
user's applications. Fuathecmore, whenever a user opens the NavUe app from their
poone, a pull prothcol will be used to retrieve and sync the latest updates from
the server.
}


\end{document}