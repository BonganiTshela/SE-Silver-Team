
\documentclass[11pt]{article}
\usepackage{makeidx}
\usepackage{multirow}
\usepackage{multicol}
\usepackage[paperwidth=595pt,paperheight=841pt,top=72pt,right=72pt,bottom=72pt,left=72pt]{geometry}

\title{Introduction}
\author{Schae Ind 14058104 }
\date{February 2017}


\begin{document}

\section{Introduction}
	\subsection{Purpose}

	The purpose of this document (Requirements Specification Document) is to outline the requirements of the NavUP application that is to be designed and built. This document will specify our requirements of the product, related to performance, external and internal interface, as well as provide the functional requirements and design constraints to be met. The requirements will help define and determine user expectations, thus allowing the team to build an effective and functional application. 

\vspace{\baselineskip}
The Requirements Specification Document will be used as an official document between the client(s) and the designers/engineers of the software to ensure that both parties understand the requirements and capabilities. This document will serve to outline the user needs and expectations for the application, as well as offer guidelines for the builders of the software. This document will help to determine what must be included in the software for it to fulfil its desired purpose.


	\subsection{Scope}

	The product to be built is called NavUP. The product will act as a navigation system for the University of Pretoria. NavUP will be a mobile application that will allow the user to navigate both indoors and outdoors at the University of Pretoria, as well as choose from a series of available routes to a set destination, based on the user’s restrictions and requirements. The application will allow the user to interact with it, much like a GPS would, thus enabling the user to view pedestrian congestion, get directions, and find and save locations. The software will also provide the user with functional, cultural and historical information about their surroundings, as well as allow them to find and navigate to a place based on said information.
\vspace{\baselineskip}

NavUP will function as a fully dedicated University of Pretoria navigation system, as well as an information application, providing valuable information on the location and history of, as well as routes to the buildings located on the campus. The software will be used to provide people who are unfamiliar with the campus with the location of specific buildings and lecture halls, and the various routes that can be used to navigate to them. NavUP will provide users with the ability to navigate on the campus based on specific restrictions or requirements, such as only showing routes where wheelchair access is available. NavUP hopes to improve class punctuality by detailing faster routes and showing pedestrian congestion. The application also hopes to minimise frustration with new students and visitors to the campus by allowing them to navigate effectively and independently across the grounds and in the buildings. NavUP will be a valuable tool in helping students and visitors alike learn about the rich cultural and architectural history of the many buildings at the University of Pretoria.

	\subsection{Definitions, Acronyms and Abbreviations}
		\subsubsection{Definitions:}
		\textbf{Requirements} – “descriptions of the services that a software system must provide” \textit{(CS2 Software Engineering note 2, 2004)} \\
		\\ \textbf{Software Requirements Specification} – Also known as a ‘Requirements Specification Document’ describes the features and intended purpose of a software application \textit{(Rouse, 2007)} \\
		\\ \textbf{Functional Requirements} – specifies something the system “should do” or a “behaviour or function” the system should have \textit{(Eriksson, 2012)} \\
		\\ \textbf{Constraint} – Something that restricts your “freedom to do what you want” \textit{(Longman Dictionary of Contemporary English, n.d.)} \\
		\\ \textbf{NavUP} – The navigational application product that is to be built


		\subsubsection{Acronyms:}
		\textbf{UP} – University of Pretoria\\
		\\ \textbf{SRS} – Software Requirements Specification


		\subsubsection{Abbreviations:}
		\textbf{App} - Aplication


	\subsection{References}

CS2 Software Engineering note 2. (2004). \textit{CS2Ah Autumn 2004}, 1. \\\\
Eriksson, U. (2012, April 5). \textit{Functional vs Non Functional Requirements. Retrieved from ReQtest}: http://reqtest.com/requirements-blog/functional-vs-non-functional-requirements/ \\\\
Longman Dictionary of Contemporary English. (n.d.). constraint. Retrieved from Longman Dictionary of Contemporary English: http://www.ldoceonline.com/dictionary/constraint \\\\
Morkel Theunissen, Vreda Pieterse, Stacey Omeleze, Emilio Singh. (2017, February). COS301 - \textit{Software Engineering. Retrieved from Computer Science}: http://www.cs.up.ac.za/courses/COS301 \\\\
Rouse, M. (2007, February). \textit{Software Requirements Specification (SRS).} Retrieved from Search Software Quality: http://searchsoftwarequality.techtarget.com/definition/software-requirements-specification





	\subsection{Overview}
This document will highlight the requirements and user expectations. It will provide a detailed overview of the product, as well as the context of the product and how it interfaces and works with other components of the entire system. The characteristics and limits of the product’s memory availability will also be discussed.
\vspace{\baselineskip}
The functions, characteristics and constraints of NavUP will be touched upon, along with specific requirements; namely, external interface, functional, performance, design and software system requirements.
\vspace{\baselineskip}
The product will be looked at in terms of what is necessary and what is possible for the specific target market it aims to satisfy.



\end{document}